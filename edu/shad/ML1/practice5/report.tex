\documentclass[12pt]{article}

\usepackage{amsfonts,amssymb}
\usepackage[utf8]{inputenc}
\usepackage[russian]{babel}
\usepackage{graphicx}
\usepackage{amsmath}
\usepackage{amsfonts}
\usepackage[ruled, lined]{algorithm2e}
\usepackage{verbatim}

\textheight=220mm
\textwidth=160mm

\newcommand{\sgn}{\operatorname{sgn}}
\newcommand{\argmax}{\operatorname{argmax}}
\newcommand{\NA}{\operatorname{NA}}
\newcommand{\OR}{\operatorname{ or }}
\newcommand{\LCS}{\operatorname{LCS}}
%\DeclareMathOperator{\sgn}{sgn}

\title{\bf Практикум к семинару № 5. \\ <<Машинное
Обучение>>}
\author{А.Е. Оразаев}
\date{}
\begin{document}

\voffset=-20mm
\hoffset=-12mm
\font\Got=eufm10 scaled\magstep2 \font\Got=eufm10

\maketitle

\section{Проклятие размерности.}
В ходе лабораторной работы было эксперементально показан эффект
"проклятия размерности".
\begin{enumerate}
    \item Загрузить пакет \verb=lestat=:
    \begin{verbatim}
    library('lestat')
    \end{verbatim}

    \item Сгенерировать точки многомерного нормального распределения:
    \begin{verbatim}
    mud = muniformdistribution(rep(-1, dim), rep(1, dim))
    points = simulate(mud, 100)
    \end{verbatim}

    \item Реализовать функцию, которая подсчитывает минимальный и
          максимальный косинус угла между сгенерированными точками:
    \begin{verbatim}
    getMinMaxAngles = function(points, index) {
      angles = data.frame(t(c(Inf, -Inf)))
      names(angles) = c("Min", "Max")
    
      if (nrow(points) == 1) {
        angles$Min = angles$Max = NaN
        return (angles)
      }
    
      elem = points[index, ]
      for (i in 1:nrow(points)) {
        if (i != index) {
          cur = points[i, ]
          cur.angle = sum(cur * elem) / (sum(cur * cur) * sum(elem * elem))
    
          if (cur.angle > angles$Max) {
            angles$Max = cur.angle
          }
    
          if (cur.angle < angles$Min) {
            angles$Min = cur.angle
          }
        }
      }

    return (angles)
    }
    \end{verbatim}

    \item Постройте график зависимости среднего минимального и максимального
          косинусов от размерности:
    \begin{center}
        \fbox{\includegraphics[width=300bp]{angles.png}}
    \end{center}

    \item Сгенерировать точки для всех размерностей:
    \begin{verbatim}
    mud = muniformdistribution(rep(-1, dim), rep(1, dim))
    points = simulate(mud, 100)
    \end{verbatim}

    \item Вычислить $ l_2 $-нормы сгенерированных точек:
    \begin{verbatim}
    getNormFunctor = function (p) {
      if (p == Inf) {
        return (function(x) { return (max(abs(x))) })
      }
      return (function(x) { return (sum(abs(x) ^ p) ^ 1/p) })
    }

    for (i in 1:length(dimensions)) {
    # ...
      norms = apply(points, 1, getNormFunctor(2))
      answersL2[i] = (max(norms) - min(norms))/max(norms)
    # ...
    }
    \end{verbatim}

    \item 8. Построить график зависимости величины $ \frac{max_i||x_i|| - min_i||x_i||}{max_i||x_i||} $.
          Проделать эксперементы для норм $ l_1 $ и $ l_{\infty} $. Постройте графики и приложите их к
          отчету.
    \begin{center}
        \fbox{\includegraphics[width=300bp]{norms.png}}
    \end{center}
\end{enumerate}

\paragraph{Выводы.} Проведенный эксперемент наглядно показывает эффект
"проклятия размерности". На графике с максимальным и минимальном
значением косинуса угла между точками можно наблюдать, что с ростом размерности
все точки становятся одинаково сильно далеки друг от друга. То есть мы теряем
"похожесть" точек, отсюда следует, что мы не сможем адекватно объеденять в группы
близкие друг к другу точки и каким-либо образом их классифицировать.

Из 2-ого графика можно сделать похожий вывод. С ростом размерности нормы дают
все более близкие значения для любых точек из распределения. В случае
$ l_{\infty} $-нормы уже на размерности 200 мы можем наблюдать, что
минимальная и максимальная норма для точек из распределения практически
одинакова. С остальными нормами ситуация получше, но они все равно не дают
различия больше чем 20\% между максимальным и минимальным значением нормы для
точек. Таким образом метрические методы классификации будут давать
неудовлетворительные результаты для пространств с большими размерностями, чтобы
этого избежать нужно прибегать к уменьшению размерности пространств.


\paragraph{Листинг. lab\_5\_angle\_todo.R}
\begingroup
    \fontsize{10pt}{12pt}\selectfont
    \verbatiminput{lab5_angle_todo.R}
\endgroup
\paragraph{Листинг. lab\_5\_norm\_todo.R}
\begingroup
    \fontsize{10pt}{12pt}\selectfont
    \verbatiminput{lab5_norm_todo.R}
\endgroup

\end{document}
