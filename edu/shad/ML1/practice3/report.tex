\documentclass[12pt]{article}

\usepackage{amsfonts,amssymb}
\usepackage[utf8]{inputenc}
\usepackage[russian]{babel}
\usepackage[dvips]{graphicx}
\usepackage{amsmath}
\usepackage{amsfonts}
\usepackage[ruled, lined]{algorithm2e}
\usepackage{verbatim}

\textheight=220mm
\textwidth=160mm

\newcommand{\sgn}{\operatorname{sgn}}
\newcommand{\argmax}{\operatorname{argmax}}
\newcommand{\NA}{\operatorname{NA}}
\newcommand{\OR}{\operatorname{ or }}
\newcommand{\LCS}{\operatorname{LCS}}
%\DeclareMathOperator{\sgn}{sgn}

\title{\bf Лабораторная работа № 3. \\ <<Машинное
Обучение.>>}
\author{А.Е. Оразаев}
\date{}
\begin{document}

\voffset=-20mm
\hoffset=-12mm
\font\Got=eufm10 scaled\magstep2 \font\Got=eufm10

\maketitle

\section{Реализация QDA.}
В ходе лабораторной работы было реализовано обучение и тестирование
квадратичного дескриминанта. Фактически пришлось внести изменения по
сравнению с готовы кодом наивного баессовского классификатора только в:
\begin{itemize}
    \item Вычисление ковариационной матрицы.
    \item Функцию \verb=qdaClassify=
\end{itemize}

Также в конце листинга можно найти обучение и клссификацию с помощью
функции \verb=qda= из пакета \textbf{MASS}.

\paragraph{Результаты.}
В итоге 4 \% ошибки на самописном \verb=qda=, абсолютно такой же
процент ошибки и для \verb=qda= из коробки.


\paragraph{Листинг.}
\begingroup
    \fontsize{10pt}{12pt}\selectfont
    \verbatiminput{qda_todo.R}
\endgroup

\end{document}
