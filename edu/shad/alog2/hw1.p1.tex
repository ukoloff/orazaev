\documentclass[12pt]{article}

\usepackage{amsfonts,amssymb}
\usepackage[utf8]{inputenc}
\usepackage[russian]{babel}
\usepackage[dvips]{graphicx}
\usepackage{amsmath}
\usepackage{amsfonts}
\usepackage[ruled, lined]{algorithm2e}

\textheight=220mm
\textwidth=160mm

\newcommand{\sgn}{\operatorname{sgn}}
\newcommand{\argmax}{\operatorname{argmax}}
\newcommand{\NA}{\operatorname{NA}}
\newcommand{\OR}{\operatorname{ or }}
\newcommand{\LCS}{\operatorname{LCS}}
%\DeclareMathOperator{\sgn}{sgn}

\title{\bf Домашнее задание по курсу \\ <<Методы
и структуры данных поиска.>>}
\author{А.Е. Оразаев}
\date{}
\begin{document}

\voffset=-20mm
\hoffset=-12mm
\font\Got=eufm10 scaled\magstep2 \font\Got=eufm10

\maketitle

\section{Задача 1-1.}
\paragraph{Описание алгоритма.}
Условимся, что будем в дальейшем индексировать строки начиная с 0,
а не с 1, дабы быть ближе к реализации. В таком случае запишем определение
префикс функции для $ s $ -- для каждого $ i \in [0, |s| - 1] $:
$$
    p[i] = \max\{j: 0 \le j < i, s[0 .. j - 1] = s[i - j .. i]\}
$$

Итак у нас есть строка $ s $ из $ 0 < N \le 1000000$ символов.
Заведем вектор $ p $ длины N для хранения значений префикс функции.
Проинициализируем $ p[0] = 0 $.

Далее для всех $ i $ в цикле от $ 1 $ до $ N - 1 $ выполняем следующие шаги:
\begin{enumerate}
    \item Если $ s[p[i - 1]] == p[i] $, то $ p[i] = p[i - 1] + 1 $, переходим
          к следующему $ i $.
    \item Если $ p[i - 1] \ne 0 \mbox{  and  } s[p[p[i - 1] - 1]] == s[i] $, то
          $ p[i] = p[p[i - 1] - 1] + 1 $, переходим к следующему $ i $.
    \item Если $ s[i] == s[0] $, то $ p[i] = 1 $, переходим к следующему
          $ i $.
    \item В противном случае $ p[i] = 0 $, переходим к следующему $ i $.
\end{enumerate}







\paragraph{Доказательство.}
% FIXME: write proof

\paragraph{Сложность}
% FIXME: write complexity

\end{document}
